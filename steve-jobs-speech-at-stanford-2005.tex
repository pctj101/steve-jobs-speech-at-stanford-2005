\documentclass[a4paper,11pt,twocolumn]{jarticle}

\setlength{\textwidth}{150mm}       % テキストの幅
\setlength{\oddsidemargin}{30mm}    % 偶数ページの左マージン
\setlength{\evensidemargin}{30mm}   % 奇数ページの左マージン
\addtolength{\oddsidemargin}{-1in}  % 元の 1in の空白を削除
\addtolength{\evensidemargin}{-1in} % 元の 1in の空白を削除

\setlength{\textheight}{232mm}      % テキストの高さ
\setlength{\headheight}{0mm}        % ヘッダの高さ
\setlength{\headsep}{0mm}           % テキストの最上部とヘッダの最下部との間隔
\setlength{\footskip}{10mm}         % テキストの最下部とフッタの最下部との間隔
\setlength{\topmargin}{35mm}        % 上のマージン
\addtolength{\topmargin}{-1in}      % 元の 1in の空白を削除

\kanjiskip=.20pt plus.05pt minus.07pt  % 全角文字の隙間の調整


\title{スタンフォード大学祝賀スピーチ}
\author{Steve Jobs}
\date{}

\begin{document}

\begin{titlepage}
\maketitle
\thispagestyle{empty}
\end{titlepage}

\section{BIRTH}

ありがとう。世界有数の最高学府を卒業される皆さんと、本日こうして晴れの門
出に同席でき大変光栄です。実を言うと私は大学を出たことがないので、これが
今までで最も大学卒業に近い経験ということになります。

本日は皆さんに私自身の人生から得たストーリーを3つ紹介します。それだけで
す。どうってことないですよね、たった3つです。最初の話は、点と点を繋ぐと
いうお話です。

私はリード大学を半年で退学しました。が、本当にやめてしまうまで18ヶ月かそ
こらはまだ大学に居残って授業を聴講していました。じゃあ、なぜ辞めたんだ?
ということになるんですけども、それは私が生まれる前の話に遡ります。

私の生みの母親は若い未婚の院生で、私のことは生まれたらすぐ養子に出すと決
めていました。育ての親は大卒でなくては、そう彼女は固く思い定めていたので、
ある弁護士の夫婦が出産と同時に私を養子として引き取ることで手筈はすべて整っ
ていたんですね。ところがいざ私がポンと出てしまうと最後のギリギリの土壇場
になってやっぱり女の子が欲しいということになってしまった。で、養子縁組待
ちのリストに名前が載っていた今の両親のところに夜も遅い時間に電話が行った
んです。「予定外の男の赤ちゃんが生まれてしまったんですけど、欲しいです
か?」。彼らは「もちろん」と答えました。

しかし、これは生みの母親も後で知ったことなんですが、二人のうち母親の方は
大学なんか一度だって出ていないし父親に至っては高校もロクに出ていないわけ
です。そうと知った生みの母親は養子縁組の最終書類にサインを拒みました。そ
うして何ヶ月かが経って今の親が将来私を大学に行かせると約束したので、さす
がの母親も態度を和らげた、といういきさつがありました。

\section{COLLEGE DROP-OUT}

こうして私の人生はスタートしました。やがて17年後、私は本当に大学に入るわ
けなんだけど、何も考えずにスタンフォード並みに学費の高いカレッジを選んで
しまったもんだから労働者階級の親の稼ぎはすべて大学の学費に消えていくんで
すね。そうして6ヶ月も過ぎた頃には、私はもうそこに何の価値も見出せなくなっ
ていた。自分が人生で何がやりたいのか私には全く分からなかったし、それを見
つける手助けをどう大学がしてくれるのかも全く分からない。なのに自分はここ
にいて、親が生涯かけて貯めた金を残らず使い果たしている。だから退学を決め
た。全てのことはうまく行くと信じてね。

そりゃ当時はかなり怖かったですよ。ただ、今こうして振り返ってみると、あれ
は人生最良の決断だったと思えます。だって退学した瞬間から興味のない必修科
目はもう採る必要がないから、そういうのは止めてしまって、その分もっともっ
と面白そうなクラスを聴講しにいけるんですからね。

夢物語とは無縁の暮らしでした。寮に自分の持ち部屋がないから夜は友達の部屋
の床に寝泊りさせてもらってたし、コーラの瓶を店に返すと5セント玉がもらえ
るんだけど、あれを貯めて食費に充てたりね。日曜の夜はいつも7マイル
(11.2km)歩いて街を抜けると、ハーレクリシュナ寺院でやっとまともなメシにあ
りつける、これが無茶苦茶旨くてね。

しかし、こうして自分の興味と直感の赴くまま当時身につけたことの多くは、あ
とになって値札がつけられないぐらい価値のあるものだって分かってきたんだね。

ひとつ具体的な話をしてみましょう。


\section{CONNECTING DOTS}

リード大学は、当時としてはおそらく国内最高水準のカリグラフィ教育を提供す
る大学でした。キャンパスのそれこそ至るところ、ポスター1枚から戸棚のひと
つひとつに貼るラベルの1枚1枚まで美しい手書きのカリグラフィ(飾り文字)が
施されていました。私は退学した身。もう普通のクラスには出なくていい。そこ
でとりあえずカリグラフィのクラスを採って、どうやったらそれができるのか勉
強してみることに決めたんです。

セリフをやってサンセリフの書体もやって、あとは活字の組み合わせに応じて字
間を調整する手法を学んだり、素晴らしいフォントを実現するためには何が必要
かを学んだり。それは美しく、歴史があり、科学では判別できない微妙なアート
の要素を持つ世界で、いざ始めてみると私はすっかり夢中になってしまったんで
すね。

こういったことは、どれも生きていく上で何ら実践の役に立ちそうのないものば
かりです。だけど、それから10年経って最初のマッキントッシュ・コンピュータ
を設計する段になって、この時の経験が丸ごと私の中に蘇ってきたんですね。で、
僕たちはその全てをマックの設計に組み込んだ。そうして完成したのは、美しい
フォント機能を備えた世界初のコンピュータでした。

もし私が大学であのコースひとつ寄り道していなかったら、マックには複数書体
も字間調整フォントも入っていなかっただろうし、ウィンドウズはマックの単な
るパクりに過ぎないので、パソコン全体で見回してもそうした機能を備えたパソ
コンは地上に1台として存在しなかったことになります。

もし私がドロップアウト(退学)していなかったら、あのカリグラフィのクラスに
はドロップイン(寄り道)していなかった。

そして、パソコンには今あるような素晴らしいフォントが搭載されていなかった。

もちろん大学にいた頃の私には、まだそんな先々のことまで読んで点と点を繋げ
てみることなんてできませんでしたよ。だけど10年後振り返ってみると、これほ
どまたハッキリクッキリ見えることもないわけで、そこなんだよね。もう一度言
います。未来に先回りして点と点を繋げて見ることはできない、君たちにできる
のは過去を振り返って繋げることだけなんだ。だからこそバラバラの点であって
も将来それが何らかのかたちで必ず繋がっていくと信じなくてはならない。自分
の根性、運命、人生、カルマ…何でもいい、とにかく信じること。点と点が自分
の歩んでいく道の途上のどこかで必ずひとつに繋がっていく、そう信じることで
君たちは確信を持って己の心の赴くまま生きていくことができる。結果、人と違
う道を行くことになってもそれは同じ。信じることで全てのことは、間違いなく
変わるんです。

\section{FIRED FROM APPLE}

2番目の話は、愛と敗北にまつわるお話です。

私は幸運でした。自分が何をしたいのか、人生の早い段階で見つけることができ
た。実家のガレージでウォズとアップルを始めたのは、私が二十歳の時でした。
がむしゃらに働いて10年後、アップルはガレージの我々たった二人の会社から従
業員4千人以上の20億ドル企業になりました。そうして自分たちが出しうる最高
の作品、マッキントッシュを発表してたった1年後、30回目の誕生日を迎えたそ
の矢先に私は会社を、クビになったんです。

自分が始めた会社だろ?どうしたらクビになるんだ?と思われるかもしれません
が、要するにこういうことです。アップルが大きくなったので私の右腕として会
社を動かせる非常に有能な人間を雇った。そして最初の1年かそこらはうまく行っ
た。けど互いの将来ビジョンにやがて亀裂が生じ始め、最後は物別れに終わって
しまった。いざ決裂する段階になって取締役会議が彼に味方したので、齢30にし
て会社を追い出されたと、そういうことです。しかも私が会社を放逐されたこと
は当時大分騒がれたので、世の中の誰もが知っていた。

自分が社会人生命の全てをかけて打ち込んできたものが消えたんですから、私は
もうズタズタでした。数ヶ月はどうしたらいいのか本当に分からなかった。自分
のせいで前の世代から受け継いだ起業家たちの業績が地に落ちた、自分は自分に
渡されたバトンを落としてしまったんだ、そう感じました。このように最悪のか
たちで全てを台無しにしてしまったことを詫びようと、デイヴィッド・パッカー
ドとボブ・ノイスにも会いました。知る人ぞ知る著名な落伍者となったことで一
時はシリコンヴァレーを離れることも考えたほどです。 

ところが、そうこうしているうちに少しずつ私の中で何かが見え始めてきたんで
す。私はまだ自分のやった仕事が好きでした。アップルでのイザコザはその気持
ちをいささかも変えなかった。振られても、まだ好きなんですね。だからもう一
度、一から出直してみることに決めたんです。

その時は分からなかったのですが、やがてアップルをクビになったことは自分の
人生最良の出来事だったのだ、ということが分かってきました。成功者であるこ
との重み、それがビギナーであることの軽さに代わった。そして、あらゆる物事
に対して前ほど自信も持てなくなった代わりに、自由になれたことで私はまた一
つ、自分の人生で最もクリエイティブな時代の絶頂期に足を踏み出すことができ
たんですね。

それに続く5年のうちに私はNeXTという会社を始め、ピクサーという会社を作り、
素晴らしい女性と恋に落ち、彼女は私の妻になりました。

ピクサーはやがてコンピュータ・アニメーションによる世界初の映画「トイ・ス
トーリー」を創り、今では世界で最も成功しているアニメーション・スタジオで
す。

思いがけない方向に物事が運び、NeXTはアップルが買収し、私はアップルに復帰。
NeXTで開発した技術は現在アップルが進める企業再生努力の中心にあります。ロ
レーヌと私は一緒に素晴らしい家庭を築いてきました。

アップルをクビになっていなかったらこうした事は何ひとつ起こらなかった、私
にはそう断言できます。そりゃひどい味の薬でしたよ。でも患者にはそれが必要
なんだろうね。人生には時としてレンガで頭をぶん殴られるようなひどいことも
起こるものなのです。だけど、信念を放り投げちゃいけない。私が挫けずにやっ
てこれたのはただ一つ、自分のやっている仕事が好きだという、その気持ちがあっ
たからです。皆さんも自分がやって好きなことを見つけなきゃいけない。それは
仕事も恋愛も根本は同じで、君たちもこれから仕事が人生の大きなパートを占め
ていくだろうけど自分が本当に心の底から満足を得たいなら進む道はただ一つ、
自分が素晴しいと信じる仕事をやる、それしかない。そして素晴らしい仕事をし
たいと思うなら進むべき道はただ一つ、好きなことを仕事にすることなんですね。
まだ見つかってないなら探し続ければいい。落ち着いてしまっちゃ駄目です。心
の問題と一緒でそういうのは見つかるとすぐピンとくるものだし、素晴らしい恋
愛と同じで年を重ねるごとにどんどんどんどん良くなっていく。だから探し続け
ること。落ち着いてしまってはいけない。

\section{ABOUT DEATH}

3つ目は、死に関するお話です。

私は17の時、こんなような言葉をどこかで読みました。確かこうです。

「来る日も来る日もこれが人生最後の日と思って生きるとしよう。そうすればい
ずれ必ず、間違いなくその通りになる日がくるだろう」。それは私にとって強烈
な印象を与える言葉でした。そしてそれから現在に至るまで33年間、私は毎朝鏡
を見て自分にこう問い掛けるのを日課としてきました。「もし今日が自分の人生
最後の日だとしたら、今日やる予定のことを私は本当にやりたいだろうか?」。
それに対する答えが“NO”の日が幾日も続くと、そろそろ何かを変える必要があ
るなと、そう悟るわけです。

自分が死と隣り合わせにあることを忘れずに思うこと。これは私がこれまで人生
を左右する重大な選択を迫られた時には常に、決断を下す最も大きな手掛かりと
なってくれました。何故なら、ありとあらゆる物事はほとんど全て…外部からの
期待の全て、己のプライドの全て、屈辱や挫折に対する恐怖の全て…こういった
ものは我々が死んだ瞬間に全て、きれいサッパリ消え去っていく以外ないものだ
からです。そして後に残されるのは本当に大事なことだけ。自分もいつかは死ぬ。
そのことを思い起こせば自分が何か失ってしまうんじゃないかという思考の落と
し穴は回避できるし、これは私の知る限り最善の防御策です。

君たちはもう素っ裸なんです。自分の心の赴くまま生きてならない理由など、何
一つない。

\section{DIAGNOSED WITH CANCER}

今から1年ほど前、私は癌と診断されました。 朝の7時半にスキャンを受けたと
ころ、私のすい臓にクッキリと腫瘍が映っていたんですね。私はその時まで、す
い臓が何かも知らなかった。

医師たちは私に言いました。これは治療不能な癌の種別である、ほぼ断定してい
いと。生きて3ヶ月から6ヶ月、それ以上の寿命は望めないだろう、と。主治医は
家に帰って仕事を片付けるよう、私に助言しました。これは医師の世界では「死
に支度をしろ」という意味のコード(符牒)です。

それはつまり、子どもたちに今後10年の間に言っておきたいことがあるのなら思
いつく限り全て、なんとか今のうちに伝えておけ、ということです。たった数ヶ
月でね。それはつまり自分の家族がなるべく楽な気持ちで対処できるよう万事しっ
かりケリをつけろ、ということです。それはつまり、さよならを告げる、という
ことです。

私はその診断結果を丸1日抱えて過ごしました。そしてその日の夕方遅く、バイ
オプシー(生検)を受け、喉から内視鏡を突っ込んで中を診てもらったんですね。
内視鏡は胃を通って腸内に入り、そこから医師たちはすい臓に針で穴を開け腫瘍
の細胞を幾つか採取しました。私は鎮静剤を服用していたのでよく分からなかっ
たんですが、その場に立ち会った妻から後で聞いた話によると、顕微鏡を覗いた
医師が私の細胞を見た途端、急に泣き出したんだそうです。何故ならそれは、す
い臓癌としては極めて稀な形状の腫瘍で、手術で直せる、そう分かったからなん
です。こうして私は手術を受け、ありがたいことに今も元気です。

これは私がこれまで生きてきた中で最も、死に際に近づいた経験ということにな
ります。この先何十年かは、これ以上近い経験はないものと願いたいですけどね。

以前の私にとって死は、意識すると役に立つことは立つんだけど純粋に頭の中の
概念に過ぎませんでした。でも、あれを経験した今だから前より多少は確信を持っ
て君たちに言えることなんだが、誰も死にたい人なんていないんだよね。天国に
行きたいと願う人ですら、まさかそこに行くために死にたいとは思わない。にも
関わらず死は我々みんなが共有する終着点なんだ。かつてそこから逃れられた人
は誰一人としていない。そしてそれは、そうあるべきことだから、そういうこと
になっているんですよ。何故と言うなら、死はおそらく生が生んだ唯一無比の、
最高の発明品だからです。それは生のチェンジエージェント、要するに古きもの
を一掃して新しきものに道筋を作っていく働きのあるものなんです。今この瞬間、
新しきものと言ったらそれは他ならぬ君たちのことだ。しかしいつか遠くない将
来、その君たちもだんだん古きものになっていって一掃される日が来る。とても
ドラマチックな言い草で済まんけど、でもそれが紛れもない真実なんです。

君たちの時間は限られている。だから自分以外の他の誰かの人生を生きて無駄に
する暇なんかない。ドグマという罠に、絡め取られてはいけない。それは他の人
たちの考え方が生んだ結果とともに生きていくということだからね。その他大勢
の意見の雑音に自分の内なる声、心、直感を掻き消されないことです。自分の内
なる声、心、直感というのは、どうしたわけか君が本当になりたいことが何か、
もうとっくの昔に知っているんだ。だからそれ以外のことは全て、二の次でいい。

\section{STAY HUNGRY, STAY FOOLISH}

私が若い頃、``The Whole Earth Catalogue(全地球カタログ)''というとんでも
ない出版物があって、同世代の間ではバイブルの一つになっていました。

それはスチュアート・ブランドという男がここからそう遠くないメンローパーク
で製作したもので、彼の詩的なタッチが誌面を実に生き生きしたものに仕上げて
いました。時代は60年代後半。パソコンやデスクトップ印刷がまだ普及する前の
話ですから、媒体は全てタイプライターとはさみ、ポラロイドカメラで作ってい
た。だけど、それはまるでグーグルが出る35年前の時代に遡って出されたグーグ
ルのペーパーバック版とも言うべきもので、理想に輝き、使えるツールと偉大な
概念がそれこそページの端から溢れ返っている、そんな印刷物でした。

スチュアートと彼のチームはこの''The Whole Earth Catalogue''の発行を何度
か重ね、コースを一通り走り切ってしまうと最終号を出した。それが70年代半ば。
私はちょうど今の君たちと同じ年頃でした。

最終号の背表紙には、まだ朝早い田舎道の写真が1枚ありました。君が冒険の好
きなタイプならヒッチハイクの途上で一度は出会う、そんな田舎道の写真です。
写真の下にはこんな言葉が書かれていました。「Stay hungry,
stayfoolish.(ハングリーであれ。馬鹿であれ)」。それが断筆する彼らが最後
に残した、お別れのメッセージでした。「Stay hungry, stay foolish.」

それからというもの私は常に自分自身そうありたいと願い続けてきた。そして今、
卒業して新たな人生に踏み出す君たちに、それを願って止みません。

\begin{center}
Stay hungry, stay foolish.
\end{center}

ご清聴ありがとうございました。

\end{document}
